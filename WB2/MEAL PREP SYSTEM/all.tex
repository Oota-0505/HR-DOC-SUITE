## **■ 0. LP概要(Summary)**

本LPは、

**「週末の作り置きを “自動で設計する” クラウド型プランニングツール」**

MEAL PREP SYSTEM は、

**週末90分でできる作り置きを “ロードマップとして自動生成” するSaaS**

の紹介・申し込みを目的としたページである。

- 作り置きの工程がワンクリックで可視化
- 保存スケジュール・買い物リストを自動計算
- 料理初心者でも迷わない動線設計
- 毎月アップデートされる“テーマ別プリセット”を利用可能
- ブラウザ上で使えるクラウド型ツール(DL不要)

本サービスはレシピ提供ツールではなく、「段取り管理 × 保存計画 × 買い物最適化」を自動化する“生活効率化プラットフォーム”です。

---

## **■ 1. サービス内容(Service Overview)**

### **● メインプラン**

**① ベーシックプラン(1,980円(税込2,178円))**

- 作業手順が毎回ブレない「一定品質のロードマップ」
- 工程図・保存マップを自動生成
- 買い物リストの自動作成
- 月替わりプリセット(低脂質 / 低糖質 / 野菜多め など)
- マイページでいつでもロードマップを閲覧・保存可能
- 忙しい日でも迷わず再開できる「途中から再開」機能
- 食材の保存期限を一覧できる「保存期限カレンダー」

**② プレミアムプラン(3,980円(税込4,278円))**

- ベーシックすべて
- 導線改善・時短ナビ機能(追加UI)
- 保存期限切れを防ぐ「保存管理アラート」
- 食材の使い忘れを防ぐ「食材ライフサイクル管理」
- チャットQ&A(一般的なアドバイスのみ)

---

### **● オプション**

- バックナンバープリセット(980円**(税込1,078円)**)
- 調理テクニック動画(500円**(税込550円)**)

---

### **● ターゲットユーザー**

- 忙しい会社員・育児中の人・自炊初心者
- 健康的な食生活を続けたいが時間がない
- コンビニ食から脱却したい
- 平日夜の調理がしんどい人

---

## **■ 2. トーン&マナー(Tone & Manner)**

本LPは以下の世界観に沿って構築する。

**「北欧 × ミニマル × 生活改善ツール」**

- 整った暮らしの世界観 + プロダクトUIの軽さ
- 白 × ベージュ × 淡いグリーン
- 角丸と余白を多用したシンプルな管理画面風レイアウト
- UIアイコンは線幅1.5pxのアウトラインアイコンを使用
- モジュール間の余白は「8px / 16px / 24px」を基準(北欧UIの規律感)
- グラデーションは禁止し、完全フラットデザインを基本にする

参考:

https://ourhome305.com

https://kurashiru.com/brand/kitchen

---

## **■ 3. カラーガイド(Color Palette)**

● メインカラー

- Sage Green(#DDE7DD)
- Warm Beige(#F3EDE4)
- Soft Olive(#C6D1BE)

● サブカラー

- Light Gray(#F6F6F6)
- White(#FFFFFF)

● アクセントカラー

- UI要素は Soft Black(#333)
- ボタンは Forest Accent(#6E8B68)
- カード輪郭は 1px #EAEAEA

---

## **■ 4. フォントガイド(Typography)**

● 見出し

- Noto Sans JP Medium
- Inter Medium(英字)

● 本文

- Noto Sans JP Regular

● 英字アクセント

- Inter Light
- Montserrat Light

---

## **■ 5. レイアウト(Layout Guideline)**

- ヘルシー系に合わせ、余白は多め
- 角丸 12px の白カードが基本
- 陰影は極薄
- 背景は白 or ごく淡い緑
- 図解・アイコンを多用(料理LPは「可視化」必須)

UIトーン

✔ カードは白・角丸16px

✔ 陰影は限りなく薄く

✔ 背景:白90%

✔ 文字色:Soft Black(#333)

---

## **■ 6. 写真・ビジュアル(Visual Guideline)**

- 「整ったキッチン」「作業台の俯瞰写真」
- 鮮やかすぎない、彩度を抑えたテイスト
- 手元・道具・食材をミニマルに表現
- ツール画面(UI)を必ず1〜3枚掲載
- スマホ・PC両方のUIモックを用意し、利用シーンを可視化
- 画面は「操作が少ない」「迷わない」ことが一目で伝わるレイアウトで撮影

※ 食材売買ではないため、食材の“販売写真”のような写実性は避ける。

---

## **■ 7. コピー(Copy Guideline)**

● トーン

- 落ち着き・丁寧・生活に寄り添う
- 時短より“習慣化”を訴求
- 食の押し売り感はゼロ

● 推奨コピー例(LP冒頭)

**「週末90分の作り置きで、平日の“自炊が整う暮らし”へ。」**

家事・仕事・育児のスキマ時間でも続けられる

やさしい作り置き“プランニングツール”を毎日使えます。

---

# **【1. ヒーローセクション】**

■ タイトル

**「平日の料理がラクになる“作り置きプラン”を、自動で設計する。」**

■ リードコピー

MEAL PREP SYSTEM は、

**作り置きの段取り・保存・買い物を “まとめて自動化” するクラウドツール** です。

- 工程マップをワンクリック生成
- 冷蔵 / 冷凍の保存ルールが自動反映
- 1週間分の食事が自然と整う

**あなたの生活に “自炊の余裕” が戻ります。**

■ 要点

- 週末90分で作れる作り置きを毎月設計
- 冷蔵・冷凍の保存ルールつき
- リマインドしやすい工程マップ
- 健康的・整った食事を取り戻す
- Before→Afterを1行で視覚化
- 視認性の高い「3アイコンのベネフィット列」を追加

■コピー案(ビジュアルも交え、視覚化を強化して視認性アップ)

**Before:平日の夜、料理がしんどい → After:帰宅して15分でごはんが整う暮らしへ。**

アイコン例:

 ✔ 自動生成される工程ロードマップ

 ✔ 保存スケジュールの最適化

 ✔ 買い物リストの自動作成

■ デザイン指示

- 木目のテーブル × ベージュの布 × 俯瞰ショット
- ファーストビュー右側にツールUIのモックアップを配置
- 週末→平日の流れが簡単にわかる画面を選ぶ
- PC版は横長UI、スマホ版は縦型UIの2構成

---

# **【共感セクション】**

■ タイトル

**「料理の“段取り”を、もっとシンプルにしたい。」**

■ 悩み

- 作り置きの工程を組み立てるのが難しい
- 迷わない“手順の地図”がほしい
- レシピを探す手間がなくなると助かる
- 食材消費の最適ルートが自動化されると安心
- 料理スキルではなく「段取りの仕組み」から整えたい
- 平日の夜に調理する余裕がない
- 冷蔵庫の管理がうまくいかない
- 時短より“迷わず進める仕組み”が必要

■ デザイン

- 3〜6個のカード並列

---

# **【MEAL PREP SYSTEM が選ばれる理由】**

### 理由①:段取りを“自動で”組み立てる

買い物 → 下ごしらえ → 調理 → 保存 → 平日の消費

をクラウド上で一括管理。

- 段取りの再現性が高まり、毎週のストレスが減る
- 保存期限切れ・使い忘れを防ぎ、無駄が減る
- 自炊が習慣化し、健康の土台が整う

---

### 理由②:保存マップを自動計算

冷蔵・冷凍の保存期限や食べ方の流れが自動反映され、

迷わず活用できる。

---

### 理由③:週末90分に最適化された“実行可能性”

レシピではなく、

**時間軸を整理したロードマップ** が主役。

---

■ デザイン指示

- 3カード横並び(スマホ縦)
- 3つのカード上に「アイコン + 一言タイトル + 詳細」(北欧UIのシンプル路線に合う構造で)

---

# **【サービス内容一覧】**

■ タイトル

**「続けやすさのために必要な機能だけ。」**

■ 内容

● プランニング機能

- ロードマップ自動生成
- 工程図のクラウド保存

● 保存・消費管理

- 保存・消費スケジュール
- 保存期限カレンダー
- プレミアム限定:時短ナビ & 改善アラート

● 買い物・プリセット

- 買い物リスト生成
- 月替わりプリセットを追加更新
- データが毎月自動でアップデート
- ロードマップ履歴をいつでも参照
- 保存・管理がクラウドのため、紛失リスクなし

※本サービスは具体的レシピや個別食事指導ではなく、**“作業の段取り”と“保存の流れ”を可視化したガイドです。**

---

# **【ご利用の流れ】**

■ タイトル

**「今日から、自炊が少しラクに。」**

STEP1|会員登録

STEP2|マイページに自動ログイン

STEP3|テーマを選択し“1秒でロードマップが生成”

STEP4|スマホ片手に工程を進められるUI

STEP5|保存期限は自動でカレンダーに反映

STEP6|毎月プリセット追加 & 自動アップデート

■ デザイン

- 横並びステップカード

---

# **【料金プラン】**

■ タイトル

**「シンプルで、続けやすい。」**

■サブコピー

- 毎月新しいプリセット・改善UIが追加されます
- ベーシックでもすべてのアップデートを自動反映

■ ベーシック

**月額 1,980円(税込2,178円)**

- ロードマップ生成
- 保存マップ
- 買い物リスト
- 月替わりプリセット
- 途中から再開できる進行ステータス管理
- 保存期限カレンダー

■ プレミアム

**月額 3,980円(税込4,278円)**

- ベーシック内容すべて
- 時短ナビ機能
- 保存管理アラート
- Q&A(一般的なガイドのみ)

■ オプション

- バックナンバー:980円**(税込1,078円)**
- HOW TO 動画:500円**(税込550円)**

■デザイン指示

- ベーシック・プレミアムの比較表(北欧系2カラムカード)を追加
- プレミアムは「Q&A特典」よりも、**“時短ナビ機能・保存管理アラート”が主役**に見えるレイアウトで違いを出す

---

# **【CTA】**

左側
■ メインコピー

**「週末90分で、平日の“整った食生活”を手に入れる。」**

■ ボタン

**「まずは作り置きプランを自動生成してみる」**  ← メインCTA

**「ツールの雰囲気だけ確認する」**               ← セカンダリ(モーダル or スクロール)

※ テーマを選ぶと、約1秒でロードマップが生成されます。

右側
お問い合わせフォーム

---

# **【会社概要】**

- 会社名:
- 代表者:
- 所在地:
- 事業内容:コンテンツ制作・教育コンテンツ配信 ほか

※Googleマップ埋める

---

# **【フッター】**

- 特定商取引法に基づく表記
- プライバシーポリシー
- 利用規約
- お問い合わせフォーム

---

# **【ログイン/マイページ仕様(共通仕様を反映)】**

■ ログイン

- /wp-login.php(デフォルトで利用)
- デザイン:ロゴ差し替えのみ可
- 非ログイン時は必ず /wp-login.php へリダイレクト

■ マイページ(MEAL PREP SYSTEM 専用)

- URL:/mypage-meal-prep
- 内容:
    - 現在のプラン選択
    - プリセット一覧
    - 保存中のロードマップ
    - バックナンバー(過去のプリセット)
    - アップデート情報
    - 保存したロードマップは月別に整理される
    - プリセットの検索・フィルター機能
    - アップデート履歴で改善内容を確認できる
    

■ 動作環境(想定)

- 対応デバイス:PC / タブレット / スマートフォン
- 推奨ブラウザ:Google Chrome / Safari / Edge 最新版
- 同一アカウントで複数デバイスからのログインが可能